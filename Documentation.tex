\documentclass[english, 12pt]{article}
\usepackage[T1]{fontenc}
\usepackage[utf8]{inputenc}
\usepackage{lmodern}
\usepackage[a4paper]{geometry}
\usepackage{babel}
\usepackage{listings}
\usepackage{hyperref}
\hypersetup{
    colorlinks=false,
    pdftitle={Overleaf Example},
    pdfpagemode=FullScreen,
}
\author{BASCOUR Gwenaël - CAUSSE Jade -  DA COSTA Yacine\\PARES Lucien - NAYET Morgan}
\title{Robot Khepera \\ Guide d'utilisation des actionneurs et capteurs}

\begin{document}
	\maketitle
	\pagebreak
	\tableofcontents
	\pagebreak
	\section{Généralités}
		Cette section est consacrée aux généralités concernant la programmation de comportements du robot.
		\subsection{Modèle de programme}
			Tous les comportements programmées sont basés sur le modèle fourni pour le robot Khepera. Le modèle s'appelle \textit{template.c} et est trouvable sur la VM fournie.
		\subsection{Compilation des programmes}
			Tous les programmes sont compilés en suivant la méthode expliquée sur le manuel d'utilisation du robot Khepera.
		\subsection{Fonctions Générales}
			Cette section est dédiée aux fonctions non-spécifiques a un fonctionnement particulier du robot, mais qui sont tout de même utilisées lors de la programmation de comportements pour celui-ci.
			\subsubsection{Changer le mode du terminal}
				Syntaxe : kb\_change\_term\_mode(int mode);
				\\
				Cette méthode sert a changer le mode du terminal afin que getChar() renvoie instantanement un caractère, sans attendre que l'utilisateur appuie sur Entrée.
				Dans cette fonction, l'argument \textit{mode} est a mettre a 0 pour avoir un terminal classique, ou a 1 pour avoir un terminal ou les fonctions nommée précédement fonctionnent.
			\subsubsection{Savoir si une touche est enfoncée}
			Syntaxe : kb\_kbhit();
			\\
			Cette méthode renvoie true si une touche est enfoncée, false sinon.
	\section{Roues (Wheels)}
		Cette section est dédiée au contrôle des déplacements du robot, grâce au controles des roues et moteurs associés.
		\subsection{Fonctions Utilisées}
			\subsubsection{Choix du mode de contrôle}
			Syntaxe : 
		
		
\end{document}